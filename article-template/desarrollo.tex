\section{Introducción}

\%El primer párrafo presenta información general sobre el tema en estudio o problema que se trata de resolver. Debe resaltar la importancia del tema tratado. y su impacto\%

\%El Segundo párrafo resalta las investigaciones previas\%

\%El tercer párrafo resalta el trabajo desarrollado , que se ha hecho (mencionar los casos) y como se hizo\%

\%El cuarto párrafo presenta la estructura del artículo. Siempre inicia con las palabras: ``El resto del artículo está organizado de la siguiente manera...'' \%


\section{Tema 1 – Realidad Aumentada}

\subsection{Definición y conceptos fundamentales}
Use automatic hyphenation and check spelling and grammar. Use high resolution (300dpi or above) figures, plots, drawings and photos for best printing result.

\subsection{Enfoques terapéuticos}
Use automatic hyphenation and check spelling and grammar. Use high resolution (300dpi or above) figures, plots, drawings and photos for best printing result.

\subsection{Desafíos y limitaciones}
Use automatic hyphenation and check spelling and grammar. Use high resolution (300dpi or above) figures, plots, drawings and photos for best printing result.


\section{Tema 2 – Marco Teórico}
\subsection{Subtema 1}
Use automatic hyphenation and check spelling and grammar. Use high resolution (300dpi or above) figures, plots, drawings and photos for best printing result.

\subsection{Subtema 2}
Use automatic hyphenation and check spelling and grammar. Use high resolution (300dpi or above) figures, plots, drawings and photos for best printing result.

\subsection{Subtema 3}
Use automatic hyphenation and check spelling and grammar. Use high resolution (300dpi or above) figures, plots, drawings and photos for best printing result.

\section{Metodología}

\subsection{Caso 1}

En el trabajo de Coronado et al. [10], se presenta el uso de la tecnología de Realidad Virtual (RV) en un sistema de control de balance para pacientes con Parkinson. El sistema utiliza el dispositivo Oculus, el cual proporciona una experiencia inmersiva a través de una simulación de bicicleta en un entorno virtual con un camino, árboles y obstáculos que deben ser evitados mediante el movimiento del cuerpo hacia la izquierda o derecha, respectivamente (Figura 4). Para la captura de los movimientos corporales, se integró el controlador Leap Motion Controller. El estudio se llevó a cabo en el Hospital Honorio Delgado, específicamente en el área de neurología.\\

Diez pacientes con enfermedad de Parkinson en estados iniciales participaron en el estudio, compuesto por cinco hombres y cinco mujeres, con edades comprendidas entre 55 y 78 años. El protocolo del estudio consistió en diez sesiones de intervención, realizadas tres veces por semana, con una duración de 30 minutos cada sesión.\\

Antes y después de las sesiones, se evaluó el desempeño de los pacientes utilizando pruebas clínicas, específicamente el Test de la Batería de Evaluación del Movimiento (BOT-2) [11] y el Test de Bloques y Cajas (Block and Box Test) [12]. Además, se recopiló la opinión de los pacientes sobre la usabilidad del sistema mediante cuestionarios de evaluación de satisfacción del usuario (User Satisfaction Evaluation Questionnaire, USEQ) [13], cuestionarios de experiencia del usuario (User Experience Questionnaire, UEQ) [14] y el Modelo de Aceptación Tecnológica (Technological Acceptance Model, TAM) [15].\\

Los resultados obtenidos mostraron una mejora significativa en la subprueba del balance del Test BOT-2 en ocho de los participantes, evidenciando los beneficios del sistema de control de balance basado en Realidad Virtual. Dos participantes no mostraron mejoría en esta área específica. En cuanto a las opiniones de los pacientes, cinco de ellos manifestaron que disfrutaron de la experiencia proporcionada por el sistema RV, mientras que los otros cinco experimentaron mareos después de utilizar el dispositivo. Los resultados del cuestionario UEQ revelaron una valoración positiva por encima de los 4 puntos, lo cual indica una alta aceptación del sistema por parte de los usuarios.


\subsection{Caso 2}
Se describe el contexto (institución y lugar), tiempo que duró la experiencia, cantidad de participantes, resultados obtenidos, software/hardware/metodología utilizados (según corresponda)

\subsection{Caso 3}
Se describe el contexto (institución y lugar), tiempo que duró la experiencia, cantidad de participantes, resultados obtenidos, software/hardware/metodología utilizados (según corresponda)

\subsection{Caso 4}
Se describe el contexto (institución y lugar), tiempo que duró la experiencia, cantidad de participantes, resultados obtenidos, software/hardware/metodología utilizados (según corresponda)
 
\section{Resultados y Discusión}
En el análisis comparativo de los cuatro casos de uso de realidad aumentada para rehabilitación en motor fino, se identificaron semejanzas y diferencias relevantes.

\subsection{Semejanzas}
Use automatic hyphenation and check spelling and grammar. Use high resolution (300dpi or above) figures, plots, drawings and photos for best printing result.

\subsection{Diferencias}
Use automatic hyphenation and check spelling and grammar. Use high resolution (300dpi or above) figures, plots, drawings and photos for best printing result.

\subsection{Discusión}
Use automatic hyphenation and check spelling and grammar. Use high resolution (300dpi or above) figures, plots, drawings and photos for best printing result.

\subsection{Propuesta}
Use automatic hyphenation and check spelling and grammar. Use high resolution (300dpi or above) figures, plots, drawings and photos for best printing result.

\section{Conclusiones}