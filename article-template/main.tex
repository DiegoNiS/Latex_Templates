\documentclass[oneside,a4]{article}                     % document properties
\usepackage[spanish, es-lcroman]{babel}                 % spanish package
\usepackage[utf8]{inputenc}                             % codificacion de caracteres utf-8
\usepackage{graphicx}                                   % Required for inserting images
\usepackage{float}                                      % float child elements
\usepackage[bottom=2.68cm,top=1.8cm,right=1.58cm,left=1.58cm]{geometry} % modify margins 
\usepackage{setspace}                                   % configure space between lines
%\usepackage{paracol}                                    % create multiple developed columns 
\usepackage{enumitem}                                   % total control of lists and its bullets
\usepackage{amsmath}                                    % math independencies
\usepackage[hidelinks]{hyperref}                        % internal links and more
\usepackage{multicol}                                   % multiple columsn
\usepackage{lipsum}                                     % lipsum random text
\usepackage{colortbl}                                   % define customize colors & table colore
\usepackage{array}                                      % customize tables  
\usepackage{rotating}                                   % allow rotate contents
\usepackage{titlesec}                                   % config auto titles
\usepackage{fancyhdr}                                   % encabezados y pies de pagina
%\usepackage{parskip}                                    % crea ambientes
\usepackage{caption}                                    % config captions
\usepackage{multirow}                                   % fusionar filas


%%%%%%%%%%%%%%%%%%%%%%%%%%%%%%%%%%%  variables  %%%%%%%%%%%%%%%%%%%%%%%%%%%%%%%
\newcommand{\itemUni}{Universidad Nacional de San Agustín}
\newcommand{\itemTheme}{Análisis Comparativo del uso de Realidad Aumentada para rehabilitación en motor fino: 4 experiencias}
\newcommand{\itemStudent}{Nombres y Apellidos Autor}
\newcommand{\itemEmail}{correo@unsa.edu.pe}
\newcommand{\itemPiePagina}{\textbf{18 LACCEI International Multi-Conference for Engineering, and Technology}: ``Engineering, Integration, and Sustainable Development'' “Hemispheric Cooperation for Competitiveness and Prosperity on a-Based Economy”, 29 July 2020, Guarena, Argentina.}


%%%%%%%%%%%%%%%%%%%%%%%%%%%%%%%%%%%  secciones y subsecciones  %%%%%%%%%%%%%%%%%%%%%%%%%%%%%%%
\titleformat{\section}[block]
  {\normalfont\small\centering}{\Roman{section}.}{1em}{\textsc}

\titleformat{\subsection}[block]
  {\normalfont\itshape}{\Alph{subsection}.}{1em}{}

\titlespacing*{\section}
  {0pt}{*1.3}{1.8mm}

\titlespacing*{\subsection}
  {0pt}{2mm}{1mm}
%%%%%%%%%%%%%%%%%%%%%%%%%%%%%%%%%%%  pies de pagina  %%%%%%%%%%%%%%%%%%%%%%%%%%%%%%%
\pagestyle{fancy}
\fancyhf{}
\renewcommand{\headrulewidth}{0pt} % Quita la línea en la cabecera
\lfoot{\footnotesize \itemPiePagina}
\rfoot{\thepage}  
%%%%%%%%%%%%%%%%%%%%%%%%%%%%%%%%%%%  titulo  %%%%%%%%%%%%%%%%%%%%%%%%%%%%%%%
\makeatletter %cambia @ para usarlo en comandos
\renewcommand{\@maketitle}{
    \begin{center}
        \begin{spacing}{2.4}
            \vspace{-0.1mm}
            \textmd{\fontsize{24}{10}\selectfont \itemTheme}\\[0.3ex]
            {\fontsize{11.3}{1}\selectfont \itemStudent$^1$,}\\[-4ex]
            {\fontsize{11.2}{1}\selectfont \textit{$^1$\itemUni, Perú, \underline{\itemEmail}}}
        \end{spacing}
    \end{center}
    \vspace{-1.06cm}
}
\makeatother %cambia @ para usarlo como caracter

\captionsetup{
  font=footnotesize,
  justification=centering,
}

\newenvironment{resumen}
{\itshape\fontsize{8}{3}\selectfont\AtBeginEnvironment{resumen}{\offinterlineskip}
\AfterEndEnvironment{resumen}{\par}}
{}
\newenvironment{tinytable}
  {\begin{center}\scriptsize}
  {\end{center}}


\newcolumntype{C}[1]{>{\centering\arraybackslash}m{#1}}
\renewcommand{\arraystretch}{1.5}
\begin{document}
    \maketitle
    \thispagestyle{fancy} % Aplicar el estilo de página en la primera página
    
    \setlength{\columnsep}{0.8cm}\begin{multicols*}{2}
        \begin{resumen}
        \textbf{Abstract} – ¿Qué se ha hecho? (Objetivo. ¿Porqué se hizo? (Problema).  Como se hizo? (Metodología). Responde a la pregunta: Porqué los resultados son útiles, importantes y a permiten ayudar a avanzar en su disciplina( Resultados)
        
        \end{resumen}
        \vspace{0.7cm}
        {\fontsize{8}{10}\selectfont\itshape 
        \textbf{Keywords-- List at most 5 key index terms here.}
        
        }\vspace{0.45cm}
        
    \section*{INTRODUCCIÓN}
\lipsum[3-5]

\clearpage
\section{SECCION 1}
\lipsum[6-7]

\section{SECCION 2}
\lipsum[8-9]

\section{SECCION 3}
\lipsum[10-12]







\clearpage
\begin{thebibliography}{X}
    \bibitem{Autor1} Complete reference here
    \bibitem{Autor2} Complete reference here
    \bibitem{Autor3} Complete reference here
\end{thebibliography}
    \vspace{1cm}
    \begin{thebibliography}{X}
    \bibitem{Manuscript} Manuscript Templates for Conference Proceedings, IEEE. \url{http://www.ieee.org/conferences_events/conferences/publishing/templates.html}
    \bibitem{King} M. King, B. Zhu, and S. Tang, ``Optimal path planning,'' Mobile Robots, vol. 8, no. 2, pp. 520-531, March 2001.
    \bibitem{Simpson} H. Simpson, Dumb Robots, 3rd ed., Springfield: UOS Press, 2004, pp.6-9.
    \bibitem{King2} M. King and B. Zhu, ``Gaming strategies,'' in Path Planning to the West, vol. II, S. Tang and M. King, Eds. Xian: Jiaoda Press, 1998, pp. 158-176.
    \bibitem{Simpson2} B. Simpson, et al, ``Title of paper goes here if known,'' unpublished.
    \bibitem{Lu} J.-G. Lu, ``Title of paper with only the first word capitalized,'' J. Name Stand. Abbrev., in press.
    \bibitem{Yorozu} Y. Yorozu, M. Hirano, K. Oka, and Y. Tagawa, ``Electron spectroscopy studies on magneto-optical media and plastic substrate interface,'' IEEE Translated J. Magn. Japan, vol. 2, pp. 740-741, August 1987 [Digest 9th Annual Conf. Magnetics Japan, p. 301, 1982]. 
    \bibitem{Young} M. Young, The Technical Writer’s Handbook, Mill Valley, CA: University Science, 1989.
\end{thebibliography}    

    \section{Anexos}

    \begin{tinytable}
        \begin{tabular}{|C{1.2cm}|p{2cm}|p{1.2cm}|p{1.5cm}|}
        \hline
            \multirow{2}{1cm}{Type Size (pts.)} & \multicolumn{3}{c|}{Apperance} \\ [10pt]\cline{2-4}
            & Regular & Bold & Italic\\\hline
            6 & Table superscripts &  & \\\hline
            8 & Section titles, references, tables, table names, table captions, figure captions, footnotes, text subscripts, and superscripts &  & \\\hline
            9 &  & Abstract, Index Terms & \\\hline
            10 & Authors affiliations, main text, equations, first letter in section titles &  & Subheading  \\\hline
            11 & Authors names &  & \\\hline
            12 & Paper title &  & \\\hline
        \end{tabular}
        \captionof{table}{This is a simple table}
    \end{tinytable}
    
    
    \begin{figure}[H]
        \centering
        \includegraphics[width=0.9\linewidth]{images/imagen.png}
        \caption{Magnetization as a function of applied field. Note caption is centered below figures, but above tables.}
    \end{figure}
    
    \end{multicols*}
\end{document}
